\documentclass{beamer}

%%%%%%%%%%%%%%%%%%%%%%%%%%%%%%%%%%%%%%%%%%%%%%%%%%%%%%%%%%%%%%%%%%%%%%%
% Definicion de paquetes
% \usepackage{minted}
\usepackage[utf8]{inputenc}
\usepackage[spanish]{babel}
\usepackage{hyperref}
\usepackage{xargs}
\usepackage[colorinlistoftodos,prependcaption,textsize=tiny]{todonotes}
\presetkeys{todonotes}{inline}{}
\usepackage{pgfpages}

%%%%%%%%%%%%%%%%%%%%%%%%%%%%%%%%%%%%%%%%%%%%%%%%%%%%%%%%%%%%%%%%%%%%%%%
% Definición de comandos
\setlength{\marginparwidth}{2cm}
% \unsure{I'm not sure}
\newcommandx{\unsure}[2][1=]{\todo[linecolor=red,backgroundcolor=red!25,bordercolor=red,#1]{#2}}
% \change{This must be changed}
\newcommandx{\change}[2][1=]{\todo[linecolor=blue,backgroundcolor=blue!25,bordercolor=blue,#1]{#2}}
% \info{Just information}
\newcommandx{\info}[2][1=]{\todo[linecolor=green,backgroundcolor=green!25,bordercolor=green,#1]{#2}}
% \improvement{THIS must be improved}
\newcommandx{\improvement}[2][1=]{\todo[linecolor=orange,backgroundcolor=orange!25,bordercolor=orange,#1]{#2}}
\newcommandx{\thiswillnotshow}[2][1=]{\todo[disable,#1]{#2}}
%%%%%%%%%%%%%%%%%%%%%%%%%%%%%%%%%%%%%%%%%%%%%%%%%%%%%%%%%%%%%%%%%%%%%%%
% Beamer
\usetheme{Madrid}
% \AtBeginEnvironment{minted}{\fontsize{12}{12}\selectfont}
% \setbeameroption{show notes on second screen=right}
% \setbeameroption{show notes}
%%%%%%%%%%%%%%%%%%%%%%%%%%%%%%%%%%%%%%%%%%%%%%%%%%%%%%%%%%%%%%%%%%%%%%%
% Título
\title[SGDI]{RabbitMQ}
\author[S. García \and M. Ruiz]{Sergio García Sánchez \and Miguel Emilio Ruiz Nieto}
%%%%%%%%%%%%%%%%%%%%%%%%%%%%%%%%%%%%%%%%%%%%%%%%%%%%%%%%%%%%%%%%%%%%%%%
%% Empieza el documento
\begin{document}
  \begin{frame}
    \titlepage
  \end{frame}

  \begin{frame}{Contenidos}
    \tableofcontents[hideallsubsections]
  \end{frame}

  \section{Introducción}
  \begin{frame}{Introducción}
    \begin{itemize}
      \item Los servicios web no tienen la capacidad de gestionar\\
      las peticiones que le llegan en un mismo momento
      \item<2-> Ejemplos:
      \begin{itemize}
        \item Web de venta de entradas ante un evento importante
        \item Servicio de videojuego online con alta demanda de
        usuarios
      \end{itemize}
      \item<3-> Como consecuencia:
      \begin{itemize}
        \item Caída del servicio
        \item Pérdida económica
        \item Pérdida de reputación
      \end{itemize}
    \end{itemize}
  \end{frame}

  \begin{frame}{Introducción}
    \onslide<+->{\begin{block}{Por tanto}
      Es necesario procesar las peticiones ``poco a poco''
    \end{block}}
    \onslide<+->{\begin{block}{Y para ello}
      Hay que implementar un \textbf{un servicio de colas}
    \end{block}}
  \end{frame}

  \begin{frame}{Introducción. Definiciones}
    \begin{itemize}
      \item Colas de mensajes
      \item Servicios de Colas
      \item Tipos
      \begin{itemize}
        \item STOMP
        \item MQTT
        \item AMQP
      \end{itemize}
    \end{itemize}

  \end{frame}

  \section{RabbitMQ}
  \begin{frame}{RabbitMQ. Introducción}

  \end{frame}

  \begin{frame}{Exchanges}

  \end{frame}

  \begin{frame}{Topics}

  \end{frame}

  \begin{frame}{Dead Lettering}

  \end{frame}

  \section{Ejemplos prácticos}
  \begin{frame}{Ejemplos prácticos}
    \begin{itemize}
      \item AMQP - Dead Lettering
      \item STOMP over WebSocket
    \end{itemize}
  \end{frame}

  \section{Conclusiones}
  \begin{frame}{Conclusiones}

  \end{frame}

  \section{Bibliografía}
  \begin{frame}{Bibliografía}

  \end{frame}
\end{document}
